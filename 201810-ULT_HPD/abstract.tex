\documentclass[a4paper]{article}
\usepackage{amssymb}
\usepackage{euscript}
\usepackage{graphicx}
\usepackage{color}
\usepackage{epsfig}
\usepackage{fullpage}

\oddsidemargin -1cm
\topmargin -1cm
\textwidth 18cm
\textheight 26cm
\pagestyle{empty}

\graphicspath{.}
\newcommand{\image}[3]{
\begin{figure}[#1]
\begin{center}
\includegraphics{full_#2.pdf}
\caption{\small#3}
\label{image:#2}
\end{center}
\end{figure}
}

\begin{document}
\title{Homogeneously precessing domain in 3He-B: a tool for studying graphene}
\author{V.~V.~Zavjalov\/\thanks{e-mail: vladislav.zavyalov@aalto.fi},  A. M. Savin, P. J. Hakonen}
\maketitle

Homogeneously precessing domain (HPD) can by observed in NMR experiments
in superfluid~$^3$He-B. HPD is characterized by coherent precession of
magnetization with large tilting angle stabilized by superfluid spin
currents. This state is a good tool for studying properties of 3He-B. We
have developed a 3He NMR setup using a dry cryostat and performed some
preliminary measurements. At present, we are preparing an experiment for
probing superfluid coherence across graphene membrane immersed in~$^3$He
by means of the HPD. I will tell about this work and about preliminary
measurements in a one-volume cell where we have found some new
interesting features of the HPD.

\image{h!}{abstr_image}{
{\bf A.} Magnet system and cell for preliminary experiments with HPD.
{\bf B.} Example of the measurement. HPD is created by continuous-wave
NMR when constant magnetic field is slowly swept down. Signal from RF
coils is recorded by oscilloscope and processed by sliding FFT. One can
see side bands near the NMR frequency which are low-frequency
oscillations of the HPD. We observe a few modes of such oscillations with
frequencies proportional to square root of the shift between NMR
frequency and Larmor frequency in the magnetic field. Also some unstable
region can bee seen at large frequency shifts. }

\end{document}
